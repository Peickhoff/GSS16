\documentclass[ngerman]{fbi-aufgabenblatt}

\usepackage[shortlabels]{enumitem}
\usepackage{listings}
% Folgende Angaben bitte anpassen

\renewcommand{\Vorlesung}{GSS}
\renewcommand{\Semester}{SoSe 2016}

\renewcommand{\Aufgabenblatt}{2}
\renewcommand{\Teilnehmer}{Chamier, Eickhoff, G�de, H�lzen, Jarsembinski}

\begin{document}
	\section{Grundlagen von Betriebssystemen}
		\begin{enumerate}
		\item[a.] Die zwei Grundaufgaben eines Betriebssystems sind die \textbf{Abstraktion von 
				Systemeigenschaften} und die \textbf{Verwaltung von Speicher}. Aus diesem Grund 
				werden Betriebssysteme auch zum einen als \"erweiterte virtuelle Machine\" gesehen, 
				die sich um die komplizierten Details der unterliegenden Maschine k�mmert und dem 
				Nutzer nur f�r ihn relevante Informationen weitergibt, sowie als Ressourcenmanager, 
				der	Speicher, Prozesse, Ger�te, etc. verwaltet.
		\item[b.] \textbf{Ressourcenmanager:} Scheduling von Prozessen oder Partitionierung von 
				Speicher.\\
				\textbf{Erweiterte virtuelle Maschine:} Verwaltung eines Dateisystems oder das 
				Bereitstellen
				einer interaktiven graphischen Benutzeroberfl�che.  
		\end{enumerate}
	\section{Prozesse und Threads}
		\begin{enumerate}
		\item[a.]
		\textbf{Programm:} Software die auf der Hardware l�uft.\\
		\textbf{Prozess:} Das "Laufen" des Programms, bzw. die einzelnen Verarbeitungsschritte, 
						vor	Abschluss dieses. Jeder Prozess l�uft auf einem Prozessor und hat 
						einen Prozessadressraum.\\
		\textbf{Thread:} "Leichtgewichtige" Prozesse, also kleinere Prozesse, von denen mehrere 
						auf einem Prozessadressraum laufen. Threads sind f�r kurzfristige Aufgaben
						geeignet.\\
		
		\item[d.] \hspace{1pt}
			\begin{figure}[!htbp]
				\includegraphics[scale=0.7]{Zustandsdiagramm.png}
			\end{figure}\\
			\textbf{Ereignisse:} a Prozess wird initialisiert, b Wert soll errechnet werden, c 
			Problem  tritt bei Berechnung auf, d Problem wurde behoben oder verworfen, e Alle 
			Berechnungen beendet, f Berechnung abgeschlossen und warten auf n�chste Anweisung	
			\newpage
			\textbf{Zust�nde:} Rechenf�hig-Der Prozess befindet sich im Hauptspeicher und erwartet
			Berechnung durch die CPU.
			Rechnend-Die CPU berechnet den Prozess. Pro CPU/Kern nur ein
			Prozess berechnet werden.\\ 
			Blockiert-Der Prozess ist aus irgendeinem Grund blockiert.Dies kann verschiedene 
			Gr�nde haben, z.B. das Warten auf ein bestimmtes Event oder Benutzereingaben.\\
			
		\end{enumerate}
		
\section{n-Adressemaschine}

\begin{figure}[h!]
\centering
\includegraphics[scale=1.0]{gssHA2-3.png}
\caption{Operanden in Baumform}
\end{figure}

\begin{lstlisting}
	Befehlsfolge			Wirkung

(1)	-->	>a1<	>H1<		H1 := a1
(2)	+ 	>a2<	>H1<		H1 := a2 + H1
(3)	-->	>a3<	>H2<		H2 := a3
(4)	/	>H1<	>H2<		H2 := H1 / H2
(5)	-->	>H2<	>H1<		H1 := H2
(6)	-->	>b2<	>H2<		H2 := b2
(7)	-	>b1<	>H2<		H2 := b1 - H2
(8)	-->	>b3<	>H3<		H3 := b3
(9)	/	>H2< 	>H3<		H3 := H2 / H3
(10)	+	>H1<	>H3<		H3 := H1 + H3
(11)	-->	>H3<	>R<		R := H3
\end{lstlisting}

Die Anzahl der Leseauftr�ge an den Speicher in der obigen Rechnung betr�gt 16. In den Speicher geschrieben wird 11 mal. Damit ergiebt sich eine Berechnungszeit von $16 \cdot 20 + 11 = 331$ Additionen.
\end{document}
