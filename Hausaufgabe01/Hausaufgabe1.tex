\documentclass[ngerman]{fbi-aufgabenblatt}

% Folgende Angaben bitte anpassen

\renewcommand{\Vorlesung}{GSS}
\renewcommand{\Semester}{SoSe 2016}

\renewcommand{\Aufgabenblatt}{1}
\renewcommand{\Teilnehmer}{Chamier, Eickhoff, G�de, H�lzen, Jarsembinski}

\begin{document}

\section*{Aufgabe 2: Schutzziele}

Abgrenzung I:\\
a)\\
Anonymit�t hei�t, dass der Sender f�r den Empf�nger, oder etwaige Dritte, unkenntlich ist. Im Gegensatz dazu wirkt Pseudonymit�t wie eine Verkleidung, das hei�t, die Verkehrsdaten des Senders werden durch einen vertraulichen Dritten verschleiert. F�r den Empf�nger sind also nur die Daten des Dritten sichtbar und nicht die des Senders direkt. Wird der gesamte Vorgang verschleiert und nicht die Daten der Beteiligten spricht man von Unbeobachtbarkeit.\\

b)\\
Der Unterschied zwischen Vertraulichkeit und Verdecktheit besteht darin, dass bei letzterem die Existenz des Datenaustauschs verschleiert wird, wohingegen bei Vertraulichkeit die Daten verschl�sselt werden, um sie vor Angreifern zu sch�tzen. Dies ist zu vergleichen mit dem Unterschied von Steganographie zu Kryptographie.\\

Abgrenzung II:\\
a)\\
Ist eine Nachricht integer, bedeutet das, dass eventuelle Besch�digungen oder andere Modifikationen vom Empf�nger erkannt werden k�nnen, w�hrend Zurechenbarkeit meint, dass der Absender tats�chlich derjenige ist, f�r den er sich ausgibt. Dies wird zum Beispiel durch digitale Signaturen gew�hrleistet.\\

b)\\
Ein Dienst ist verf�gbar, wenn er zu einem gew�nschten Zeitpunkt genutzt werden kann, wohingegen er auch erreichbar sein kann ohne die gew�nschte Fuktion bereitzustellen.\\


Techniken:\\
Anonymit�t: Fraglich im Internet mit IPs, Lokal vllt. ohne Benutzerkonto
Pseudonymit�t: Proxy
Vertraulichkeit: Verschl�sselung (z.B. RSA, AES, WhatsApp-Ende-zu-Ende-Verschl�sselung)
Verdecktheit: Steganographie (Nachrichten in JPEGs o.�.)
Integrit�t: Hashsummen, Parit�tsbits 
Zurechenbarkeit: Digitale Signatur
Verf�gbarkeit: Cloud
Erreichbarkeit: Lokal,Internet,Infrarot


\section*{Aufgabe 3: Angreifermodell}

2.\\
Rolle: Au�enstehender(nur mit karte oder nur mit pin), andere Benutzer des Automaten
Verbreitung: Ablesen an Tastatur, Aufs�tze am Automaten, Diebstahl
Verhalten: Ablesen(beobachtend), Aufs�tze(aktiv), Diebstahl(Aktiv)
Rechenkapazit�t: Ablesen(0), Aufs�tze(0),Diebstahl(nur karte - enorm)

\section*{Aufgabe 5: Passwortsicherheit}

Brute-Force-Angriff:
$62^8\div1000000\div60\div60\div24=2527,1$ Tage

Bei beliebiger Passwortl�nge mit Zahlen:
$(10^1 + ... + 10^{16}) \div 1000000 \div 3600 \div 24 = 128601$ Tage
\end{document}
