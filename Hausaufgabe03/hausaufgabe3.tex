\documentclass[ngerman]{fbi-aufgabenblatt}

\usepackage[shortlabels]{enumitem}
\usepackage{blockgraph}
% Folgende Angaben bitte anpassen

\renewcommand{\Vorlesung}{GSS}
\renewcommand{\Semester}{SoSe 2016}

\renewcommand{\Aufgabenblatt}{3}
\renewcommand{\Teilnehmer}{Chamier, Eickhoff, G�de, H�lzen, Jarsembinski}

\begin{document}
\section{Scheduling-Algorithmen}
a)\\
\begin{figure}[h!]
\centering
\includegraphics[scale=0.4]{graph-41a.png}
\end{figure}

b)\\
\begin{figure}[h!]
\centering
\includegraphics[scale=0.33]{graph-41b.png}
\end{figure}

\section{Echtzeit und Mehrprozessor-Scheduling}
a)\\
Die Summe aller Bedienzeiten inklusive Initialisierungs- und Prozesswechselzeiten ergibt $(1+1)+(1+3)+(1+1) = 8$ und ist damit gr��er als die Periodendauer von $A_2$. Die Deadline des zweiten Auftrags kann also nie eingehalten werden.\\

b)\\
\begin{figure}[h!]
\centering
\includegraphics[scale=0.41]{graph-42b.png}
\end{figure}

c)\\
\begin{figure}[h!]
\centering
\includegraphics[scale=0.5]{graph-42c.png}
\end{figure}


\end{document}
