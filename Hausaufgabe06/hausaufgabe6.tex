\documentclass[ngerman]{fbi-aufgabenblatt}

\usepackage[shortlabels]{enumitem}
\usepackage{listings}
% Folgende Angaben bitte anpassen

\definecolor{dkgreen}{rgb}{0,0.6,0}
\definecolor{gray}{rgb}{0.5,0.5,0.5}
\definecolor{mauve}{rgb}{0.58,0,0.82}


\lstset{frame=tb,
  language=Java,
  aboveskip=3mm,
  belowskip=3mm,
  showstringspaces=false,
  columns=flexible,
  basicstyle={\small\ttfamily},
  numbers=left,
  numberstyle=\tiny\color{gray},
  keywordstyle=\color{blue},
  commentstyle=\color{dkgreen},
  stringstyle=\color{mauve},
  breaklines=true,
  breakatwhitespace=true,
  tabsize=3,
}

\renewcommand{\Vorlesung}{GSS}
\renewcommand{\Semester}{SoSe 2016}

\renewcommand{\Aufgabenblatt}{3}
\renewcommand{\Teilnehmer}{Chamier, Eickhoff, G�de, H�lzen, Jarsembinski}

\begin{document}
\section{Rechnersicherheit}
\subsection{Zugangs- und Zugriffskontrolle}
\begin{enumerate}
\item
Die \textbf{Zugangskontrolle} dient der Identifikation beim Zugang auf Betriebsmittel und verbietet dabei unbefugten Zugang. Es wird nur mit berechtigten Partnern kommuniziert.\\
Die \textbf{Zugriffskontrolle} hingegen gibt einem identifizierten Kommunikationspartner nur den Zugriff auf Objekte, f�r die er Zugriffsrechte hat.\\
\item
Eine Zugangskontrolle ohne interne Zugriffkontrolle ist durchaus sinvoll, da zwar keine Unterscheidung im Zugriff auf Objekte gemacht wird, sehr wohl jedoch der Zugang zum System insgesamt reguliert wird.\\
\item
F�r eine Zugriffskontrolle muss der Nutzer des Systems identifiziert sein, um ihm Zugriffsrechte zuordnen zu k�nnen.\\
\end{enumerate}

\subsection*{1.4 Realisierung eines Online-Ticket-Verkaufs}
\begin{enumerate}
\item
\textbf{Rolle}: Au�enstehender (Kunde)\\
\textbf{Verbreitung}: Onlinetransaktion(1), Ticket�bermittlung(2), andere Kunden(3), Alte Codes von Tickets auswerten(4)\\
\textbf{Verhalten}: (1,2) passiv beobachtend, (3,4) aktiv beobachtend\\
\textbf{Rechenkapazit�t}: beschr�nkt.\\
\item
Die Barcodes werden auf einem System im Kino erzeugt und auf dieses System kann man nur lokal zugreifen.\\
Die Barcodes werden pseudozuf�llig erzeugt, beliebig h�ufiges hashen, codes sind Unikate.\\
\\
Kauft nun also jemand ein Ticket online, so wird auf besagtem System ein Barcode erzeugt und an diesen per Mail versandt. (Davon ausgehend, dass die Mail des Kunden sicher ist).\\
\\
Kommt also jetzt besagter Kunde zum Kino, so wird der Barcode vom lokalen Server abgeglichen.\\
\end{enumerate}

\section{Rechnersicherheit}
1.\\

3.\\
Zuerst testet der Angreifer auf L�nge, d.h. wenn das Passwort pr�fen l�nger dauert, ist die L�nge erreicht.\\
Danach testet man Zeichen f�r Zeichen durch, von links nach rechts, jedes mal wenn ein weiteres Zeichen richtig ist, braucht das Programm l�nger.\\

\section{Real-World-Brute-Force Angriff}
Der Zugangscode f�r das svs-Submission-Tool besteht aus einem festen Header (GSS16), den wir als bekannt voraussetzen, einer Zahl von 0-9 und vier f�nfstelligen Zeichenketten aus Gro�buchstaben und Zahlen (36 Zeichen).\\
Die Wahrscheinlichkeit einen der 100 vergebenen Codes zu treffen ist also:
\begin{equation}
P=\frac{100}{10 \cdot 36^{20}} \approx 0
\end{equation}
Ananas begeht Sudoku, damit das Sushi schwitzt!
\end{document}
